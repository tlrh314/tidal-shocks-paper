\documentclass{article}

\usepackage{graphicx}
\usepackage{amsmath}
\usepackage{amssymb}
\usepackage{natbib}
\usepackage{hyperref, nameref}
\hypersetup{colorlinks=true,linkcolor=blue,citecolor=blue,filecolor=blue,urlcolor=blue}

\begin{document}

\title{Internal structure of (globular) star clusters}
\author{T.L.R. Halbesma}

\maketitle


\section{Timescales}\label{sec:timescales}
\begin{itemize}
    \item \nameref{sec:t_cross}
    \item \nameref{sec:t_relax}
    \item \nameref{sec:t_equipartition}
\end{itemize}


\subsection{Crossing Time}\label{sec:t_cross}
Copy-Pasta from \citet{2010gnbs.book.....A} \\

``The crossing time is undoubtedly the most intuitive time-scale relating to 
self-gravitational systems. For a system in approximate dynamical equilibrium
it is defined by 
\begin{align}
t_{\text{cr}} &= 2 R_V / \sigma \quad ,
\end{align}
\noindent where $R_V$ is the virial radius, obtained 
from the potential energy by $R_V = G N^2 \bar{m}^2 / 2 |U|$, and $\sigma$ is 
the rms velocity dispersion. In a state of approximate equilibrium, 
$\sigma^2 \approx GN \bar{m} / 2 R_V$, which gives 

\begin{align}
t_{\text{cr}} &\approx 2 \sqrt{2} (R_V^3 / G N \bar{m})^{1/2} \quad ,
\end{align}
\noindent with $\bar{m}$ the mean mass, or alternatively 
$t_{\text{cr}} = G (N \bar{m})^{5/2} / (2 |E|)^{3/2}$ from $E = \frac{1}{2}U$.
Unless the total energy is positive, any significant deviation frmo overall 
equilibrium causes a stellar system to adjust globally on this timescale which
is also comparable to the free-fall time.


\subsection{Relaxation Time}\label{sec:t_relax}
``The subject of relaxation time is fundamental and was mainly formulated by
Rosseland [1928], Ambartsumian [1938, 1985], Spitzer [1940] and
Chandrasekhar [1942]. The classical expression is given by

\begin{align}
    T_E &= \frac{1}{16} \left(\frac{3 \pi}{2}\right)^{1/2} 
        \left(\frac{NR^3}{Gm}\right)^{1/2} \frac{1}{\ln{0.4N}}
\end{align}
\noindent where R is the size of the homogeneous system [Chandrasekhar, 1942].
For the purposes of star cluster dynamics, the half-mass relaxation time is
perhaps more useful since it is not sensitive to the density profile.

Following Spitzer [1987], it is defined by\footnote{Also see Spitzer \& Hart [1971a]
for an alternative derivation.}
\begin{align}
    t_{\text{rh}} &= 0.138 \left(\frac{Nr_h^3}{Gm}\right)^{1/2} \frac{1}{\ln{(\gamma N)}},
\end{align}
\noindent where $r_h$ is the half-mass radius and $\Lambda = \gamma N$ is the
argument of the Coulomb logarithm. Formally this factor is obtainer by integrating
over all impact parameters in two-body encounters, with a historical value of
$\gamma = 0.4$.''


\subsection{Equipartition Time}\label{sec:t_equipartition}
Analysis of a two-component system dominated by light particles gave rise to 
the equipartition time for kinetic energy [Spitzer, 1969]
\begin{align}
    t_{\text{eq}} &= \frac{( \bar{v_1^2} + \bar{v_2^2})^{3/2}}{8(6 \pi)^{1/2} G^2 \rho_{01} m_2 \ln N_1}
\end{align}



\newpage
\section{Radii}\label{sec:radii}
\begin{itemize}
    \item \nameref{sec:r_vir}
    \item \nameref{sec:r_cl}
\end{itemize}

\subsection{Virial Radius}\label{sec:r_vir}
\begin{align}
    R_V = G N^2 \bar{m}^2 / 2 |U|
\end{align}

\subsection{Close Encounter}\label{sec:r_cl}
\begin{align}
    R_{\text{cl}} = 2 G \bar{m} / sigma^2
\end{align}
\noindent which takes the simple form $R_{\text{cl}} \approx 4 R_V / N$ at
equilibrium.



\bibliographystyle{mnras}
\bibliography{tidalshocks}

\end{document}
