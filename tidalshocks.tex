% mnras_template.tex
%
% LaTeX template for creating an MNRAS paper
%
% v3.0 released 14 May 2015
% (version numbers match those of mnras.cls)
%
% Copyright (C) Royal Astronomical Society 2015
% Authors:
% Keith T. Smith (Royal Astronomical Society)

% Change log
%
% v3.0 May 2015
%    Renamed to match the new package name
%    Version number matches mnras.cls
%    A few minor tweaks to wording
% v1.0 September 2013
%    Beta testing only - never publicly released
%    First version: a simple (ish) template for creating an MNRAS paper

%%%%%%%%%%%%%%%%%%%%%%%%%%%%%%%%%%%%%%%%%%%%%%%%%%
% Basic setup. Most papers should leave these options alone.
\documentclass[a4paper,fleqn,usenatbib]{mnras}

% MNRAS is set in Times font. If you don't have this installed (most LaTeX
% installations will be fine) or prefer the old Computer Modern fonts, comment
% out the following line
\usepackage{newtxtext,newtxmath}
% Depending on your LaTeX fonts installation, you might get better results with one of these:
%\usepackage{mathptmx}
%\usepackage{txfonts}

% Use vector fonts, so it zooms properly in on-screen viewing software
% Don't change these lines unless you know what you are doing
\usepackage[T1]{fontenc}
\usepackage{ae,aecompl}


%%%%% AUTHORS - PLACE YOUR OWN PACKAGES HERE %%%%%

% Only include extra packages if you really need them. Common packages are:
\usepackage{graphicx}	% Including figure files
\usepackage{amsmath}	% Advanced maths commands
\usepackage{amssymb}	% Extra maths symbols

\usepackage{todonotes}
\graphicspath{{img/}}   % Set image path

%%%%%%%%%%%%%%%%%%%%%%%%%%%%%%%%%%%%%%%%%%%%%%%%%%

%%%%% AUTHORS - PLACE YOUR OWN COMMANDS HERE %%%%%

% Please keep new commands to a minimum, and use \newcommand not \def to avoid
% overwriting existing commands. Example:
%\newcommand{\pcm}{\,cm$^{-2}$}	% per cm-squared
\newcommand{\Sun}[0]{\ensuremath{_{\odot}}}
\renewcommand{\deg}{\ensuremath{^{\circ}}}

%%%%%%%%%%%%%%%%%%%%%%%%%%%%%%%%%%%%%%%%%%%%%%%%%%

%%%%%%%%%%%%%%%%%%% TITLE PAGE %%%%%%%%%%%%%%%%%%%

% Title of the paper, and the short title which is used in the headers.
% Keep the title short and informative.
\title[GCs in MWPotential2014]{The effect of the Galactic tidal field onto the 
distribution of stars in the Milky Way globular clusters}

% The list of authors, and the short list which is used in the headers.
% If you need two or more lines of authors, add an extra line using \newauthor
\author[T. L. R. Halbesma et al.]{\parbox[t]{\textwidth}{
    Timo L. R. Halbesma$^{1}$\thanks{E-mail: Halbesma@MPA-Garching.MPG.DE},
    Simon D. M. White$^{1}$
} \vspace{10pt} \\
% List of institutions
$^{1}$ Max-Planck-Institut f\"ur Astrophysik, Karl-Schwarzschild-Str. 1,
    85741 Garching, Germany \\
}

% These dates will be filled out by the publisher
\date{Accepted XXX. Received YYY; in original form ZZZ}

% Enter the current year, for the copyright statements etc.
\pubyear{2020}

% Don't change these lines
\begin{document}
\label{firstpage}
\pagerange{\pageref{firstpage}--\pageref{lastpage}}
\maketitle

% Abstract of the paper
\begin{abstract}
\end{abstract}

% Select between one and six entries from the list of approved keywords.
% Don't make up new ones.
\begin{keywords}
methods: numerical -- galaxies: formation -- galaxies: star clusters: general.
\end{keywords}

%%%%%%%%%%%%%%%%%%%%%%%%%%%%%%%%%%%%%%%%%%%%%%%%%%

%%%%%%%%%%%%%%%%% BODY OF PAPER %%%%%%%%%%%%%%%%%%


\section{Introduction}
\label{sec:introduction}

We investigate whether the distribution of stars in the MW GCs is consistent
with their orbits in the Galactic potential. 


\todo[inline]{Effect of tidal shocks on clusters}
\begin{itemize}
\item \citet{2018arXiv181200014W} 
\item \citep{1985ApJ...295..374A, 1986ApJ...307...97A}
\item \citet{1958ApJ...127...17S}
\item \citet{1987degc.book.....S}
\item \citet{2006MNRAS.371..793G}
\item \citet{2011MNRAS.414.1339K}
\item \citet{2016MNRAS.463L.103G}
\end{itemize}

\todo[inline]{
``By defining a tidal heating parameter \citep{2003ApJ...582..141G}, cluster's
mass loss history can be estimated for any known tidal history.''

\url{https://arxiv.org/pdf/1911.01548.pdf}
``If the dynamical evolution timescale is larger than the orbital period, these 
stripped stars remain for some time on approximately the mean orbit of the 
progenitor (Lynden-Bell \& Lynden-Bell 1995)''
}



We summarise observations of the distribution of stars in the MW GCs and the
initial conditions for the orbit integrations in section~\ref{sec:observations}.
We present our method in section~\ref{sec:simulations}, 
results in section~\ref{sec:results}, 
discussion in section~\ref{sec:discussion} 
and conclude in section~\ref{sec:conclusion}.

\section{Observations}
\label{sec:observations}
Recent observational catalogues now provide full 6D phase space information for
the Milky Way (MW) globular cluster (GC) system. \citet{2019MNRAS.482.5138B,2020IAUS..351..451H}
published\footnote{\url{https://people.smp.uq.edu.au/HolgerBaumgardt/globular},
version Jan. 2020} mean proper motions (PMs) and line-of-sight (radial) velocities 
(RVs) of 159 MW GCs and compiled velocity dispersion profiles of 141 MW GCs. The 
authors combined PMs from HST \citep{2014ApJ...797..115B,2015ApJ...803...29W}
and \textit{Gaia} DR2 measurements with RVs from MUSE integral-field spectroscopy
\citep{2018MNRAS.473.5591K} and from the WAGGS survey and archival AAT data 
\citep{2020MNRAS.492.3859D}. 
\todo[inline]{
The details of the N-body models can be found in \citet{2017MNRAS.464.2174B}, 
and \citet{2018MNRAS.478.1520B}.
Details on the stellar mass functions can be found in \citet{2017MNRAS.471.3668S}.
Finally, the rotation profiles of the clusters were derived in \citet{2019MNRAS.485.1460S}.
}

In addition, \citet{2019MNRAS.485.4906D} studied the distribution of stars in 81 MW GCs using
observations of their radial number density profiles stitched together from 
\textit{Gaia} DR2 \citep{2016A&A...595A...1G,2016A&A...595A...2G,2018A&A...616A...2L}
observations of the outer regions 
together with 26 Hubble Space Telescope (HST) star count profiles \citep{2013ApJ...774..151M} 
plus ground-based surface brightness profiles \citep{1995AJ....109..218T,1995AJ....109.1912T}
of the inner regions. 
\todo[inline]{
Decide which references suffice for DR2, so either drop or include the following:
``parallax and PM down to G=21 (Evans+ 2018, Lindegren+ 2018, Riello+ 2018),
RVs from \textit{Gaia} RV spectrometer (RVS) spectrograph (Cropper+ 2018, Sartoretti+ 2018),
dust maps (Schlegel+ 1998) /w coefficients from  Schlafly \& Finkbeiner 2011,
extinction from Schlegel+ 1998, Harris+ 1996, 2010 ed''
}
\noindent The authors fit the lowered isothermal models of \citet{1966AJ.....71...64K}, 
\citet{1975AJ.....80..175W}, \citet{2015MNRAS.454..576G,2018MNRAS.474.3997G}
and the spherical models of star clusters with potential escapers
\citep[SPES,][]{2017MNRAS.466.3937C,2019MNRAS.487..147C} using the \textsc{limepy} code
\citep{2015MNRAS.454..576G,2018MNRAS.474.3997G}.
\todo[inline]{
Paraphrase and add to introduction:
``Davoust (1977) showed that the King and Wilson models are members of a general
family of models, Gomez-Leyton \& Velazquez (2014) generalised the model to
non-integer terms. Gieles \& Zocchi (2015) added radial velocity isotropy
(Eddington 1915, Michie 1963), multiple mass components (Da Costa \& Freeman 1976,
Gunn \& Griffin 1979)''
}

\citet{2018MNRAS.474.2479B} conducted a study of MW GC mass evolution along
their orbit in the MWPotential2014, using \textsc{Galpy} for orbit integrations
coupled to the \textsc{emacss} \citep{2012MNRAS.422.3415A,2014MNRAS.442.1265A,
2014MNRAS.437..916G} code for the GC mass evolution. The initial conditions
for the orbit realisations were compiled from the literature before \textit{Gaia} DR2.
The authors calculate the initial mass of MW GCs, mass loss, and Jacobi radii. \\




\section{Simulations}
\label{sec:simulations}

\todo[inline]{Various options}
\begin{itemize}
    \item \textsc{nbody6} \citep{2003gnbs.book.....A,2010gnbs.book.....A}
    \item \textsc{nbody6tt} \citep{2011MNRAS.418..759R,2015ascl.soft02010R,2015MNRAS.448.3416R}
    \begin{itemize}
        \item ``Mode A: the tidal information is extracted from a galaxy or cosmology 
            simulation, in the form of tidal tensors, along one orbit. This method is 
            described in \citep{2011MNRAS.418..759R}''
        \item ``Mode B: the user defines a numerical function which takes position and time 
            as arguments, and returns the galactic potential. This method is described in
            \citep{2015MNRAS.448.3416R}''
    \end{itemize}
    \item ICs with \textsc{McLuster} \citep{2011ascl.soft07015K,2011MNRAS.417.2300K}
    \item \textsc{galpy} \citep{2015ApJS..216...29B}
    \begin{itemize}
        \item \textsc{galpy.potential.ttensor} \citep{2019MNRAS.488.5748W}
        \item \textsc{galpy.potential.rtide} \citep{2019MNRAS.488.5748W}
        \item \textsc{galpy.potential.to\_amuse} \citep{2019arXiv191001646W}
        \item \textsc{MWPotential2014} \citep{2015ApJS..216...29B}
    \end{itemize}
    \item \textsc{amuse} \citep{2009NewA...14..369P,2013CoPhC.184..456P,
        2013A&A...557A..84P,2018araa.book.....P}
    \item \textsc{Gadget-2} \citep{2001NewA....6...79S,2005MNRAS.364.1105S}
    \item \textsc{Arepo} \citep{2010MNRAS.401..791S,2011MNRAS.418.1392P,
        2013MNRAS.432..176P,2016MNRAS.455.1134P}
    \item \textsc{Arepo-public} \citep{2019arXiv190904667W}
\end{itemize}


\section{Results}
\label{sec:results}


\section{Discussion}
\label{sec:discussion}


\section{Conclusions}
\label{sec:conclusions}

The last numbered section should briefly summarise what has been done, and describe
the final conclusions which the authors draw from their work.

\section*{Acknowledgements}
The research was conducted using the \textsc{Python} \citep{python}
programming language with the \textsc{IPython} \citep{2007CSE.....9c..21P} environment.
We used the \textsc{NumPy} \citep{2011CSE....13b..22V}, 
\textsc{SciPy} \citep{2020SciPy-NMeth},
\textsc{galpy}\footnote{https://github.com/jobovy/galpy} \citep{2015ApJS..216...29B}, 
\textsc{pynbody} \citep{2013ascl.soft05002P}
\textsc{Astropy}\footnote{https://www.astropy.org} \citep{2013A&A...558A..33A, 2018AJ....156..123A}, 
\textsc{Astroquery} \citep{2019AJ....157...98G},
\textsc{emcee} \citep{2013PASP..125..306F}, and \textsc{corner} \citep{corner} packages.
Furthermore, we use the \textsc{amuse} \citep{2009NewA...14..369P,
2013CoPhC.184..456P,2013A&A...557A..84P,2018araa.book.....P} framework.
Plots were generated using \textsc{Matplotlib} \citep{2007CSE.....9...90H}.

%%%%%%%%%%%%%%%%%%%%%%%%%%%%%%%%%%%%%%%%%%%%%%%%%%

%%%%%%%%%%%%%%%%%%%% REFERENCES %%%%%%%%%%%%%%%%%%

% The best way to enter references is to use BibTeX:

\bibliographystyle{mnras}
\bibliography{tidalshocks}

%%%%%%%%%%%%%%%%%%%%%%%%%%%%%%%%%%%%%%%%%%%%%%%%%%

%%%%%%%%%%%%%%%%% APPENDICES %%%%%%%%%%%%%%%%%%%%%

\appendix

\section{Simulations}
\begin{table*}\caption{Stability and runtime checks for different parameters}
\begin{tabular}{l l c c c c c}
    \hline
    GC & model & N & $\epsilon$ & dt & $\eta$ & seed \\
    \hline
    NGC 104 & King & $10^3$ & 0.1 & 0-0.01 & 0.025 & 1337 \\
    NGC 104 & King & $10^4$ & 0.1 & 0-0.01 & 0.025 & 1337 \\
    NGC 104 & King & $10^4$ & 1.0 & 0-0.01 & 0.025 & 1337 \\
    NGC 104 & King & $10^4$ & 0.01 & 0-0.01 & 0.025 & 1337 \\
    NGC 104 & King & $10^5$ & 0.1 & 0-0.01 & 0.025 & 1337 \\
    \hline
    \hline
\end{tabular}
\end{table*}

\begin{table*}\caption{Emcee fits for fun}
\begin{tabular}{l c c c c}
NGC 104
NGC 288
NGC 362
NGC 1261
Pal 1
NGC 1851
NGC 1904
NGC 2298
NGC 2419
NGC 2808
NGC 3201
NGC 4147
NGC 4590
NGC 5024
NGC 5053
NGC 5139
NGC 5272
NGC 5286
NGC 5466
NGC 5634
NGC 5694
IC 4499
NGC 5824
NGC 5897
NGC 5904
NGC 5986
NGC 6093
NGC 6121
NGC 6101
NGC 6144
NGC 6139
NGC 6171
NGC 6205
NGC 6229
NGC 6218
NGC 6235
NGC 6254
NGC 6266
NGC 6273
NGC 6284
NGC 6293
NGC 6341
NGC 6325
NGC 6333
NGC 6352
NGC 6366
NGC 6362
NGC 6388
NGC 6402
NGC 6397
NGC 6426
NGC 6496
NGC 6539
NGC 6541
IC 1276
NGC 6569
NGC 6584
NGC 6624
NGC 6626
NGC 6637
NGC 6652
NGC 6656
Pal 8
NGC 6681
NGC 6715
NGC 6717
NGC 6723
NGC 6752
NGC 6779
NGC 6809
Pal 11
NGC 6864
NGC 6934
NGC 6981
NGC 7006
NGC 7078
NGC 7089
NGC 7099
Pal 12
NGC 7492
    \hline
    \hline
\end{tabular}
\end{table*}

%%%%%%%%%%%%%%%%%%%%%%%%%%%%%%%%%%%%%%%%%%%%%%%%%%


% Don't change these lines
\bsp	% typesetting comment
\label{lastpage}
\end{document}

% End of mnras_template.tex
